\documentclass[a4paper,portrait]{article}
\usepackage[margin=15mm]{geometry}
\usepackage[ngerman]{babel}
\usepackage{tikz}
\usepackage{pifont}
\usepackage{graphicx}

\begin{document}

\pgfmathsetmacro{\cardroundingradius}{4mm}
\pgfmathsetmacro{\striproundingradius}{3mm}
\pgfmathsetmacro{\cardwidth}{6}
\pgfmathsetmacro{\cardheight}{8}
\newcommand{\stripcolor}{cyan}
\pgfmathsetmacro{\stripwidth}{0}
\pgfmathsetmacro{\strippadding}{0}
\pgfmathsetmacro{\imagewidth}{\cardwidth}

\pgfmathsetmacro{\titlewidth}{0.18*\cardwidth}
\pgfmathsetmacro{\titleheight}{1.015*\cardheight}



\newcommand{\striptext}{INTER ARMA \rotatebox[origin=c]{-90}{\ding{49}}}
\pgfmathsetmacro{\textpadding}{0.3}
\newcommand{\titlecaption}{LATIN}
\newcommand{\topcontent}{\textit{''Inter Arma Enim Silent Leges''}}
\newcommand{\bottomcaption}{Inter Arma}
\newcommand{\bottomcontent}{In times of war, the law falls silent.\\ \tikz{\fill[even odd rule] (0,0) circle (0.3) (-0.2,-0.2) rectangle (0.2,0.2);}}
\pgfmathsetmacro{\ruleheight}{0.1}
\newcommand{\stripfontsize}{\Huge}
\newcommand{\captionfontsize}{\LARGE}
\newcommand{\textfontsize}{\large}
\newcommand{\titlefontsize}{\large}


\def\shapeCard{(0,0) rectangle (\cardwidth,\cardheight)}

\tikzset{
    %   runde Ecken für die Karten
    cardcorners/.style={
        rounded corners=0.2cm
    },
	%   Bild für das Kartenmotiv
    cardimage/.style={
        path picture={
            \node[below=-1.5mm] at (0.5*\cardwidth,\cardheight) {
                \includegraphics[width=\imagewidth cm]{#1}
            };
        }
    },
}

\newcommand{\CreateCard}{\draw[rounded corners=\cardroundingradius] (0,0) rectangle (\cardwidth,\cardheight);}

\newcommand{\SetBackground}[1]{\draw[cardcorners, cardimage=#1] \shapeCard;}

\newcommand{\SetTitle}[1]{
\node[text width=3.75cm, below right, inner sep=0] at (\titlewidth, \titleheight)
    {   \begin{center}
    		{#1}
		\end{center}     };
}

\newcommand{\SetCastingTime}[1]{
\node[text width=3.75cm, below right, inner sep=0] at (-0.23*\titlewidth, 0.873*\titleheight)
    {   \begin{center}
    		{#1}
		\end{center}     };
}

\newcommand{\SetRange}[1]{
\node[text width=3.75cm, below right, inner sep=0] at (2.25*\titlewidth, 0.873*\titleheight)
    {   \begin{center}
    		{#1}
		\end{center}     };
}

\newcommand{\SetComponents}[1]{
\node[text width=2.5cm, below right, inner sep=0] at (0.3*\titlewidth, 0.78*\titleheight)
    {   \begin{center}
    		{#1}
		\end{center}     };
}

\newcommand{\SetDuration}[1]{
\node[text width=3.75cm, below right, inner sep=0] at (2.25*\titlewidth, 0.78*\titleheight)
    {   \begin{center}
    		{#1}
		\end{center}     };
}

\newcommand{\SetText}[1]{
\node[text width=4.75cm, below right, inner sep=0] at (0.045*\cardwidth, 0.61*\cardheight)
    { {#1} };
}


\begin{tikzpicture}
	\CreateCard
	\SetBackground{spellbkg.jpg}
	\SetTitle{Healing Word}
	\SetCastingTime{1 bonus action}
	\SetRange{60 feet}
	\SetComponents{V}
	\SetDuration{Instantaneous}
	\SetText{A creature you can see regains hit points equal to 1d4 + your spellcasting ability modifier. This spell has no effect on undead or constructs.

\textbf{At Higher Levels:} The healing increases by 1d4 for each slot level above 1st.}
\end{tikzpicture}

\end{document}